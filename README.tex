% Options for packages loaded elsewhere
\PassOptionsToPackage{unicode}{hyperref}
\PassOptionsToPackage{hyphens}{url}
%
\documentclass[
]{article}
\usepackage{lmodern}
\usepackage{amssymb,amsmath}
\usepackage{ifxetex,ifluatex}
\ifnum 0\ifxetex 1\fi\ifluatex 1\fi=0 % if pdftex
  \usepackage[T1]{fontenc}
  \usepackage[utf8]{inputenc}
  \usepackage{textcomp} % provide euro and other symbols
\else % if luatex or xetex
  \usepackage{unicode-math}
  \defaultfontfeatures{Scale=MatchLowercase}
  \defaultfontfeatures[\rmfamily]{Ligatures=TeX,Scale=1}
\fi
% Use upquote if available, for straight quotes in verbatim environments
\IfFileExists{upquote.sty}{\usepackage{upquote}}{}
\IfFileExists{microtype.sty}{% use microtype if available
  \usepackage[]{microtype}
  \UseMicrotypeSet[protrusion]{basicmath} % disable protrusion for tt fonts
}{}
\makeatletter
\@ifundefined{KOMAClassName}{% if non-KOMA class
  \IfFileExists{parskip.sty}{%
    \usepackage{parskip}
  }{% else
    \setlength{\parindent}{0pt}
    \setlength{\parskip}{6pt plus 2pt minus 1pt}}
}{% if KOMA class
  \KOMAoptions{parskip=half}}
\makeatother
\usepackage{xcolor}
\IfFileExists{xurl.sty}{\usepackage{xurl}}{} % add URL line breaks if available
\IfFileExists{bookmark.sty}{\usepackage{bookmark}}{\usepackage{hyperref}}
\hypersetup{
  hidelinks,
  pdfcreator={LaTeX via pandoc}}
\urlstyle{same} % disable monospaced font for URLs
\usepackage{longtable,booktabs}
% Correct order of tables after \paragraph or \subparagraph
\usepackage{etoolbox}
\makeatletter
\patchcmd\longtable{\par}{\if@noskipsec\mbox{}\fi\par}{}{}
\makeatother
% Allow footnotes in longtable head/foot
\IfFileExists{footnotehyper.sty}{\usepackage{footnotehyper}}{\usepackage{footnote}}
\makesavenoteenv{longtable}
\usepackage{graphicx}
\makeatletter
\def\maxwidth{\ifdim\Gin@nat@width>\linewidth\linewidth\else\Gin@nat@width\fi}
\def\maxheight{\ifdim\Gin@nat@height>\textheight\textheight\else\Gin@nat@height\fi}
\makeatother
% Scale images if necessary, so that they will not overflow the page
% margins by default, and it is still possible to overwrite the defaults
% using explicit options in \includegraphics[width, height, ...]{}
\setkeys{Gin}{width=\maxwidth,height=\maxheight,keepaspectratio}
% Set default figure placement to htbp
\makeatletter
\def\fps@figure{htbp}
\makeatother
\setlength{\emergencystretch}{3em} % prevent overfull lines
\providecommand{\tightlist}{%
  \setlength{\itemsep}{0pt}\setlength{\parskip}{0pt}}
\setcounter{secnumdepth}{-\maxdimen} % remove section numbering

\author{}
\date{}

\begin{document}

\hypertarget{ascco2}{%
\section{ASC\_CO2}\label{ascco2}}

Simulation of amplitude of seasonal cycle (ASC) of CO\textsubscript{2}
using P model.

\hypertarget{scheme-and-procedure}{%
\subsection{Scheme and procedure}\label{scheme-and-procedure}}

The simulation of CO\textsubscript{2} ASC spans from 1901 to 2016 on a
global scale (0.5*0.5 degree), assuming vegetation carbon pool and soil
carbon pool is in equilibrium in 1901, and calculate changes in these
pools and consequently the ecosystem respiration due to environmental
changes, and finally the amplitude of CO\textsubscript{2} ASC using
atmospheric transport modelling (TM3).

In this simulation, gross primary production (GPP) is first quanified
using \href{https://github.com/davidorme/pyrealm/tree/master}{Pyrealm
package} based on P model. Using biomass production efficiency (BFE),
GPP is partitioned into autotrophic respiration (Ra) and net primary
production (NPP), which contributes to vegetation biomass carbon. Using
vegetation turnover rate, vegetation biomass carbon is converted to soil
carbon , and then heterotrophic respiration can be calculated from
labile soil carbon, a fraction of soil carbon that is accessible to
decomposition.

A flowchart of the simulation is provided below. The detailed procedure
and data inputs used in the simulation is in later section.

\begin{figure}
\centering
\includegraphics[width=9.6875in,height=\textheight]{image-20210506122407917.png}
\caption{}
\end{figure}

Specifically in each year, vegetation biomass carbon (\(C_{veg}\)) and
soil carbon (\(C_{soil}\)) are calculated recursively as:

\[C_{veg}(i) = C_{veg}(i-1)+NPP(i)-C_{veg}(i) \times k_{veg}\]

\[C_{soil}(i) = C_{soil}(i-1)+C_{veg}(i) \times k_{veg}-C_{labC}(i)\times k_{s}\]

Assuming at steady state in 1901 and an invariant vegetation turnover
rate \(k_{veg}\) is calculated:

\[k_{veg}=\frac{NPP(1901)}{Cveg(1901)}\]

Labile soil carbon (\(C_{labC}\)) is calculated as
\(C_{labC} = \alpha C_{soil}\), \(\alpha\) is the function of
environmental variables (TBC).

\hypertarget{biomass-production-efficiency-bfe}{%
\subsubsection{Biomass production efficiency
(BFE)}\label{biomass-production-efficiency-bfe}}

Net primary production (NPP) are calculated using biomass production
efficiency, the ratio of NPP/GPP: NPP = GPP × \(\sum\)BFE; then
autotrophic respiration (Ra) can be calcualted as Ra = GPP - NPP.

BFE is a constant value for each plant functional types (PFT) except for
forest. For forest, BFE = 0.19+0.006×MAT-0.00038×age+6.8E-5×TAP
+0.0039×\textbar lat\textbar. MAT stands for mean annual temperature,
while TAP stands for total annual precipitation. The BFE value for other
PFT are derived from He et al. (2020)\footnote{He, Y, Peng, S, Liu, Y,
  et al. Global vegetation biomass production efficiency constrained by
  models and observations. \emph{Glob Change Biol}.; 26: 1474-- 1484
  (2020). \url{https://doi.org/10.1111/gcb.14816}} and Campioli et al.
(2015)\footnote{Campioli, M., Vicca, S., Luyssaert, S. \emph{et al.}
  Biomass production efficiency controlled by management in temperate
  and boreal ecosystems. \emph{Nature Geosci} \textbf{8,} 843--846
  (2015). \url{https://doi.org/10.1038/ngeo2553}}, and is shown in the
table

\begin{longtable}[]{@{}ll@{}}
\toprule
Plant functional type (PFT) & Biomass production efficiency
(BFE)\tabularnewline
\midrule
\endhead
Grassland & 0.45\tabularnewline
Cropland & 0.55\tabularnewline
Tundra & 0.45\tabularnewline
Savanna & 0.47\tabularnewline
Shrubland & 0.47\tabularnewline
\bottomrule
\end{longtable}

\hypertarget{soil-heterotrophic-respiration-rate}{%
\subsubsection{Soil heterotrophic respiration
rate}\label{soil-heterotrophic-respiration-rate}}

Soil heterotrophic respiration rate is calculated as
\(k_s = f(T) \times f(M)\), while f(T) and f(M) is soil heterotrophic
respiration affected by temperature and soil moisture respectively.

The temperature function of soil heterotrophic respiration uses the
\(Q_{10}\), but instead of a simple \(Q_{10}\) with single air
temperature, we used a depth-resolved f(T) averaging f(T) at every 10cm
over the full 0-1 m depth interval following Kovern et al.
2017\footnote{Koven, C., Hugelius, G., Lawrence, D. \emph{et al.} Higher
  climatological temperature sensitivity of soil carbon in cold than
  warm climates. \emph{Nature Clim Change} \textbf{7,} 817--822 (2017).
  \url{https://doi.org/10.1038/nclimate3421}}, to account for the
vertical variation in soil climate. Temperature at each depth is
calculated following Campbell and Norman (1998)\footnote{Campbell, G.
  S., \& Norman, J. \emph{An introduction to environmental biophysics}.
  Springer Science \& Business Media. (2012).} as follows:

\[T_{z,t} = T_{mean}+A_0exp(-z/d)sin(\omega t-z/d)\]

where \(T_{mean}\) is the annual mean air temperature, \(A_0\) isthe
annual temperature amplitude (half of the difference between the coldest
and warmest month air temperature), \(\omega=2 \pi / \tau\) with
\(\tau\)=12 months. d is damping depth given by d=2k/\(\omega\) where k
is the abundance-weighted thermal diffusivity of soil components. Here k
(m\textsuperscript{2} month\textsuperscript{-1} ) is fixed for each soil
texture type: k for soil organic matter, clay, silt and sand is 0.368,
0.815, 0.946 and 1.76 respectively.

Based on reference f(T) at 15 degree celsius, the monthly f(T) is
calculated using

\[f(T) = \sum_0^{z}f(15)Q_{10}^{(T_{z}-15)/10}\]

Where \(Q_{10}\) is fixed at 1.5.

The moisture function of soil respiration rate followed Yan et al.
(2018)\footnote{Yan, Z., Bond-Lamberty, B., Todd-Brown, K.E. \emph{et
  al.} A moisture function of soil heterotrophic respiration that
  incorporates microscale processes. \emph{Nat Commun} \textbf{9,} 2562
  (2018). \url{https://doi.org/10.1038/s41467-018-04971-6}} as follows:

\[f(M) = \left\{
\begin{array}{lcl}
\frac{K_\theta+\theta_{op}}{K_\theta+\theta}(\frac{\theta}{\theta_{op}})^{1+an_s},  &   & {\theta<\theta_{op}}\\
(\frac{\phi-\theta}{\phi-\theta_{op}})^b,     &      & {\theta \geq \theta_{op}}\\
\end{array} \right.\]

Where \(\theta\) is soil moisture (m\textsuperscript{3}
m\textsuperscript{-3}), \(\theta_{op}\) is optimum water content given
by soil porosity \(\phi\) as \(\theta_{op}=0.65 \phi\). \(K_{\theta}\)
is the moisture constant; \(n_s\) is saturation exponent dependent on
soil structure and texture; b is the O\textsubscript{2} supply
restriction factor. Here the value for \(K_{\theta}, n_s\) and b is
fixed at 0.1, 2 and 0.75 respectively.

Monthly soil respiration is calculated as above and aggregated to obtain
annual soil respiration.

\hypertarget{forcing-data}{%
\subsection{Forcing data}\label{forcing-data}}

GPP is forced by meterological data from CRU4.04, which covers monthly
mean air temperature, minimum and maximum temperature and vapour
pressure. Solar radiation are derived from
\href{https://catalogue.ceh.ac.uk/documents/31dd5dd3-85b7-45f3-96a3-6e6023b0ad61}{WFD}
(1901-1978) combined with
\href{https://cds.climate.copernicus.eu/cdsapp\#!/dataset/10.24381/cds.20d54e34?tab=overview}{WFDE5}
(1979-2016) with WFD corrected to match with WFDE5. fAPAR data were
downloaded from
\href{https://drive.google.com/drive/folders/0BwL88nwumpqYaFJmR2poS0d1ZDQ}{GIMMS
3g fAPAR}; as remote sensing data of vegetation cover is not avilable
before 1982, fAPAR in 1982 was used for period 1901-1981.

Meterological data derived from CRU4.01 was used in SPLASH v2.0,
including precipitation, temperature and cloud cover, which then
coverted to solar radiation.

Forest age map used to quantify forest biomass production efficiency is
downloaded at \href{https://doi.pangaea.de/10.1594/PANGAEA.897392}{GFAD
v1.1}, and land cover used to calculate NPP and Ra are derived from
\href{https://www.atmos.illinois.edu/~meiyapp2/datasets.htm}{ISAM-HYDE}
(1901-2010) and MODIS land cover product
\href{https://lpdaac.usgs.gov/products/mcd12c1v006/}{MCD12C1 v006}
(2011-2016). For consistency, initial global carbon pool data of
vegetation and soil were also derived from ISAM model in TRENDY v8.

Soil property data, including fraction of soil texture type, porosity,
and fraction of organic matter were derived from
\href{https://data.isric.org/geonetwork/srv/eng/catalog.search\#/metadata/d9eca770-29a4-4d95-bf93-f32e1ab419c3}{ISRIC-WISE
global dataset (v3.0)} to quantify soil respiration rate.

\hypertarget{reference}{%
\subsection{Reference}\label{reference}}

\end{document}
